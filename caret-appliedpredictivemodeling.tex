\documentclass[main.tex]{subfiles}
\begin{document}
%----------------------------------------------------------------------------------------%
\section{AppliedPredictiveModeling}
This package contains several data set and A few functions from Kuhn's and Johnson's Springer book \emph{'Applied Predictive Modeling'}.

\subsection{Fuel Economy Data Set}
The http://fueleconomy.gov website, run by the U.S. Department of Energy’s Office of Energy
Efficiency and Renewable Energy and the U.S. Environmental Protection Agency, lists different
estimates of fuel economy for passenger cars and trucks. For each vehicle, various characteristics
are recorded such as the engine displacement or number of cylinders. Along with these values,
laboratory measurements are made for the city and highway miles per gallon (MPG) of the car.

\subsection{Hepatic Data Set}
This data set was used to develop a model for predicting compounds’ probability of causing hepatic
injury (i.e. liver damage). This data set consisted of 281 unique compounds; 376 predictors were
measured or computed for each. The response was categorical (either "None", "Mild" or "Severe"
),and was highly unbalanced.

This kind of response often occurs in pharmaceutical data because companies steer away from
creating molecules that have undesirable characteristics. Therefore, well-behaved molecules often
greatly outnumber undesirable molecules. The predictors consisted of measurements from 184
biological screens and 192 chemical feature predictors. 
The biological predictors represent activity
for each screen and take values between 0 and 10 with a mode of 4. The chemical feature predictors
represent counts of important sub-structures as well as measures of physical properties that are
thought to be associated with hepatic injury.

\end{document}
