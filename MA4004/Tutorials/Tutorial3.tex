\documentclass[12pt]{article}
\usepackage{amsmath}
%\usepackage[paperwidth=21cm, paperheight=29.8cm]{geometry}
\usepackage[angle=0,scale=1,color=black,hshift=-0.4cm,vshift=15cm]{background}
\usepackage{multirow}
\usepackage{enumerate}

%\SetBgScale{1}
%\SetBgAngle{0}
%\SetBgColor{black}
%\SetBgContents{\rule{1pt}{30cm}}
%\SetBgHshift{-8.4cm}
%
%\backgroundsetup{contents={
%\begin{tabular}{c|c}
%\hspace{2cm} & \\[0.7cm]
%& {\bf Statistics for Computing ------ Lecture 1 ------ Solutions} \\[0.3cm]
%%\hline
%\hspace{2cm} & \hspace{18.5cm} \\ [28cm]
%\end{tabular}}}

\backgroundsetup{contents={
{\bf \centering Statistics for Computing ------------------ Tutorial 3 --------------------------- Questions} }}


\setlength{\voffset}{-3cm}
\setlength{\hoffset}{-2cm}
\setlength{\parindent}{0cm}
\setlength{\textheight}{27cm}
\setlength{\textwidth}{17cm}

\pagestyle{empty}



\begin{document}

\subsection*{Question 1}
Assume that there are three different routes to get to a particular location: $R_1$, $R_2$ and $R_3$. You take $R_1$ 75\% of the time, $R_2$ 20\% of the time and $R_3$ the rest of the time. If you take $R_1$, there is a 90\% chance that you will be on time; if you take $R_2$, there is a 50\% chance that you will be on time and, if you take $R_3$, there is a 70\% chance that you will be on time. \\[0.1cm]
Let $T$ represent on time.\\[-0.2cm]

{\bf(a)} If $T$ represents ``on time'', what notation would we use for ``late''? \quad {\bf(b)} What is the value of $\Pr(R_1 \cap R_2)$? \quad {\bf(c)} Calculate the probability of being on time. \quad {\bf(d)} \emph{Given that} you \emph{are} on time, calculate the probabilities of having used each of the routes. \quad {\bf(e)} Given that you are late, what is the probability that you used $R_1$?


\subsection*{Question 2}
Assume that 40\% of emails you receive can be classed as ``spam''. Your spam filter places 90\% of spam emails in the spam folder and 1\% of non-spam emails in the spam folder.\\[-0.2cm]


{\bf(a)} Using appropriate notation, convert the above into mathematical statements. \quad {\bf(b)} What proportion of your emails go to the spam folder? \quad {\bf(c)} Given that an email is in the spam folder, what is the probability that it is in fact spam? \quad {\bf(d)} Given that the email is \emph{not} in the spam folder, what is the probability that it is spam?


\subsection*{Question 3}
Let's assume you have four shirts (green, red, brown, black), two jackets (blue, black) and two pairs of trousers (brown, black).\\[-0.2cm]

{\bf(a)} How many outfits have you got altogether? \quad {\bf(b)} What if the shirt must be red? \quad {\bf(c)} What if the shirt must be green or black? \quad {\bf(d)} What if the shirt must be green and the jacket must be blue? \quad {\bf(e)} What if no item is to be black? \quad {\bf(f)} What if at least one item must be black? \quad {\bf(g)} What if the shirt and jacket can be the same colour but the trousers must be a different colour?


\subsection*{Question 4}
You must create a password of length 5 using the following characters:
$\{a,b,c,d,1,2,3,4\}$.\\[-0.2cm]

{\bf(a)} How many possible passwords are there? \quad {\bf(b)} What if you only use letters? \quad {\bf(c)} What if first character cannot be a number? \quad {\bf(d)} How many passwords contain \emph{at least one} number? \quad {\bf(e)} What if each character must be different?


\subsection*{Question 5}
In your favourite RPG game, let's assume that in selecting your character there are 5 character classes and 2 genders. Let's also assume there are 3 levels of difficulty for this game.\\[-0.2cm]

{\bf(a)} How many possible ways can you play this game? \quad {\bf(b)} What if you always choose the ``warrior'' class? \quad {\bf(c)} What if you always choose a female character? \quad {\bf(d)} What if you always play on the highest difficulty setting? \quad {\bf(e)} Let's assume the game has a two-player mode. How many possible ways can you play this game? (hint: you cannot play the game on different difficulty levels). {\bf(f)} What if your friend chooses a different character class to you?


\subsection*{Question 6}
Assume that you are going to an exam and you can only bring 3 items. You have the following items: $\{\text{mobile phone, } \text{pen, } \text{ruler, } \text{calculator, } \text{laptop, } \text{apple} \}$.\\[-0.2cm]

{\bf(a)} In order to make your decision, you first \emph{arrange} these 6 items on your desk. How many possible arrangements are there? \quad {\bf(b)} How many possible groups of three items can you bring with you? \quad {\bf(c)} What if you decide that the pen is essential? \quad {\bf(d)} What if the pen is essential and you also decide that you won't bring an apple or a laptop?


\subsection*{Question 7}
A team of 5 people is required to perform a particular task. We are selecting from a group of 7 women and 3 men.\\[0.2cm]
How many selections are there:\\[-0.2cm]

{\bf(a)} Altogether? \quad {\bf(b)} If one of the men is an expert and must be on the team? \quad {\bf(c)} If two of the individuals do not get along and cannot be on the team together? \quad {\bf(d)} If the group must contain 3 women and 2 men? {\bf(e)} If the group must contain more women than men? \quad {\bf(f)} If the group must contain more men than women?


\end{document} 