\documentclass[12pt]{article}
\usepackage{amsmath}
\usepackage{multirow}
\usepackage{enumerate}
\usepackage{graphicx}
\usepackage{changepage}
\usepackage[all]{xy}
\usepackage{tikz}
\usetikzlibrary{shapes}


\setlength{\voffset}{-3cm}
%\setlength{\hoffset}{-2cm}
\setlength{\parindent}{0cm}
\setlength{\textheight}{26cm}
%\setlength{\textwidth}{14cm}


\begin{document}

\section*{MA4413 Assignment \qquad{\small(10\% of Module)}}
\noindent\rule{\linewidth}{1pt}
\quad\\[-0.5cm]

\subsection*{Details}
\begin{itemize}
\item This assignment must be your own work. Evidence of copying will result in a score of zero.
\item The deadline for this assignment is 25/11/2014:
\begin{itemize}
\item Hard copy to be submitted during the lecture time.
\item Electronic copy and \texttt{R} script file containing all code to be submitted using the ``Assignments'' section on sulis.
\end{itemize}
%\item A number of students may be randomly selected to discuss their assignment (not in front of the class).
\item Each student has their own set of data. You will find this in the \mbox{``MA4413-R-Assignment-Datasets''} excel file (row containing your ID).
\end{itemize}
\subsection*{Report}
\begin{itemize}
\item Results must be accompanied by concise description. Do not waffle.
\item Your report will be a word document:
\begin{itemize}
\item Text size: 12pt.
\item Main Font: arial, calibri or times new roman.
\item Font for \texttt{R} output: \texttt{courier}.
\end{itemize}
\end{itemize}
\subsection*{Hypothesis Tests}
\begin{itemize}
\item When you see the phrase ``carry out the test'', you are required to:
\begin{itemize}
\item carry out the test using \texttt{R};
\item copy the \texttt{R} output into your report (use \texttt{courier} font);
\item clearly state the null and alternative hypotheses;
\item provide conclusion based on the p-values and confidence intervals.
\end{itemize}
\end{itemize}
\subsection*{Independent Research}
\begin{itemize}
\item There are two items highlighted as being ``independent research''. These items have not been covered in the lectures; it is up to you to find out how to do them yourself.
\end{itemize}

\vfill\noindent\rule{\linewidth}{1pt}


\newpage


\section*{Comparing Two CPU Designs \hfill{\scriptsize \bf (30 marks)}}
\noindent\rule{\linewidth}{1pt}
\quad\\[-0.5cm]

A manufacturer of CPUs wants to compare two designs in terms of their typical operating temperature (degrees Celsius). A sample of each type was selected and the temperature was recorded after 1 hour of normal use.\\[0.2cm]

You have been assigned the task of analysing this data. Based on the analysis, you will decide whether or not there is a difference between the two designs and, if there is a difference, state which CPU design is superior (i.e., lower temperature is better).\\[0.2cm]

The measurements for CPU 1 and CPU 2 can be found in the  ``MA4413-R-Assignment-Datasets'' excel file. Each student has their own data. Find the row in the excel file containing your ID.\\[0.2cm]

The marking scheme is as follows:\\
{\footnotesize(complete the tasks in the order shown below)}

\begin{enumerate}[1)]
\item[] {\bf Presentation} \hfill{\scriptsize \bf (5 marks)}
\begin{itemize}
\item There are 5 marks going for overall presentation of your report.
\end{itemize}
\item {\bf Graphical and Numerical Summaries} (Lecs 1 and 2) \hfill{\scriptsize \bf (8 marks)}
\begin{itemize}
\item Copy the data into an \texttt{R} script file and name the data vectors \texttt{cpu1} and \text{cpu2} respectively.
\item Plot histograms for each CPU type. Define the breakpoints in \texttt{hist} yourself so that there are 5 classes that span the data. The code \texttt{seq(min(x), max(x), length=6)} achieves this where \texttt{x} is a vector of data.
\item Plot the two boxplots on the same graph.
\item Calculate the mean, standard deviation, quartiles, IQR, minimum and maximum temperature for both CPUs. Present these in a table with two columns - one for each CPU - and round all numbers to two decimal places.
\item Comment on all of the above output with reference to the shape of the distributions, centre and spread etc. but {\bf do not waffle}. This must be concise and to the point.
\end{itemize}
\item {\bf Check for Normality of Data} (Lec 10) \hfill{\scriptsize \bf (4 marks)}
\begin{itemize}
\item Use Q-Q plots to determine whether or not the two datasets are approximately normally distributed (also refer to the histograms and boxplots).
\item \emph{(independent research)} There also exists a hypothesis test of normality called the Shapiro-Wilk test. Carry out this test in \texttt{R}. %Copy the \texttt{R} output into your report, clearly state the null and alternative hypotheses and provide your conclusion based on the p-value {\bf(use statistical language)}.
    \\[0.3cm]
\end{itemize}
\item {\bf Difference Between Population Means} (Lecs 13,14,15,16) \hfill{\scriptsize \bf (9 marks)}
\begin{itemize}
\item While the summaries from part (1) above are useful for describing a \emph{sample} of data, we cannot make statements about the \emph{whole population} without constructing confidence intervals and performing hypothesis tests. %In particular, we wish to compare the population means.
\item Carry out the F-test. %Copy the \texttt{R} output into your report, clearly state the null and alternative hypotheses and provide your conclusion based on the p-value and confidence interval {\bf(use statistical language)}.
\item Carry out the t-test for two independent samples (assuming equal variances if possible). %Copy the \texttt{R} output into your report, clearly state the null and alternative hypotheses and provide your conclusion based on the p-value and confidence interval  {\bf(use statistical language)}.
\item \emph{(independent research)} When the samples are small, the t-test requires the assumption that the data is normally distributed. If this is not the case \emph{non-parametric methods} can be used. For comparing two independent groups, the Wilcoxon rank-sum test (also known as the Mann-Whitney U test) can be used. Carry out this test in \texttt{R}.
    {\footnotesize(note: do not confuse with the Wilcoxon signed-rank test for paired data)}
\end{itemize}
\item {\bf Final Summary} \hfill{\scriptsize \bf (4 marks)}
\begin{itemize}
\item Briefly summarise the main results of your analysis. Also, provide your final conclusion in non-statistical language. {\bf Do not waffle}. A few key sentences is sufficient (and no more than half a page).
\end{itemize}
\end{enumerate}

\noindent\rule{\linewidth}{1pt}


%\noindent\rule{\linewidth}{1pt}
%\quad\\[-0.5cm]
%
%
%
%\begin{enumerate}[1)]
%\item {\bf Graphical and Numerical Summaries} (Lecs 1 and 2) \hfill{\scriptsize \bf (8 marks)}
%\begin{itemize}
%\item Copy the data into an \texttt{R} script file and name the data vectors \texttt{cpu1} and \text{cpu2} respectively.
%\end{itemize}
%\end{enumerate}



\end{document} 