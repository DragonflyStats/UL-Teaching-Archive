\section{The \texttt{train} Function}
This function sets up a grid of tuning parameters for a number of classification and regression routines, fits each model and calculates a resampling based performance measure.

%-------------------------------------------------------------------------------------------------%

train can be used to tune models by picking the complexity parameters that are associated with the optimal resampling statistics. For particular model, a grid of parameters (if any) is created and the model is trained on slightly different data for each candidate combination of tuning parameters. Across each data set, the performance of held-out samples is calculated and the mean and standard deviation is summarized for each combination. The combination with the optimal resampling statistic is chosen as the final model and the entire training set is used to fit a final model.

%-------------------------------------------------------------------------------------------------%


One of the primary tools in the package is the train function which can be used to

\begin{itemize}
\item evaluate, using resampling, the effect of model tuning parameters on performance
\item choose the “optimal” model across these parameters
\item estimate model performance from a training set
\end{itemize}

\end{document}
