\documentclass[]{article}

\usepackage{framed}
\usepackage{amsmath}
\begin{document}
	
	\Large
	%===============================================================%
\newpage
\subsection*{Question 1}

Q1. (a) The manager of a shopping centre wants to profile Sunday shoppers at the centre.

Three questions from the questionnaire given to a random sample of shoppers are as follows:

\begin{framed}
Q1. How often do you visit this shopping centre on Sundays?

Every week ?

Once a month ?

A few times a year ?

Q2. How much do you spend on average per visit?

Q3. Rate your opinion of the facilities offered by this centre on a scale of 0 to 10 where 0=very poor and 10=excellent.
\end{framed}

Classify the data generated for each of the above questions by data type and scale of

measurement. (6 marks)

(b)What is a simple random sample? What are the advantages of taking a simple random sample? (2 marks)

\newpage

(c) The sales of a product are normally distributed with a mean of 200 units per week and a standard deviation of 40 units per week.

\begin{itemize}
\item[(i)] What is the probability that more than 250 units will be sold in any given week?

\item[(ii)] What is the probability that between 220 and 260 units will be sold in any given week?

\item[(iii)] To have a 98\% probability that the company will have sufficient stock to cover the weekly demand, how many units should be produced?

\item[(iv)] Use a control chart to identify which of the following weekly sales figures (in units) were unusually high or unusually low: 280, 330, 50, 310, 285, 65.

\end{itemize}
%---------------------------------%
\newpage
(c) A mobile phone company wanted to find out how its customers rated the customer service offered by its call centre operators. A list of all customers was obtained and 500 customers were randomly selected from the list. These 500 customers were posted a questionnaire asking them to rate customer service on a scale of 0 to 10. Customers who responded were given €10 credit on their account. 250 customers responded and gave a mean rating of 5 for customer service.

\begin{itemize}
\item[(i)] For this example, identify the population, the sampling frame, the sample and the variable measured.

\item[(ii)] What is the main parameter of interest in this example?

\item[(iii)] What is the best estimate of this parameter?

\item[(iv)] Describe the potential bias in this example.
\end{itemize}
(7 marks)

\end{document}