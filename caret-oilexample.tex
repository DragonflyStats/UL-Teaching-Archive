\documentclass[caret-main.tex]{subfiles}
\begin{document}
\section{Oil}
% Fatty Acid Composition Data

Brodnjak-Voncina et al. (2005) describe a set of data where seven fatty acid compositions were used to classify commercial oils as either pumpkin (labeled A), sunflower (B), peanut (C), olive (D), soybean (E), rapeseed (F) and corn (G). There were 96 data points contained in their Table 1 with known results. The breakdown of the classes is given in below:

\begin{framed}
\begin{verbatim}
data(oil)
dim(fattyAcids)
[1] 96  7
table(oilType)
\end{verbatim}
\end{framed}
\begin{verbatim}
oilType
 A  B  C  D  E  F  G 
37 26  3  7 11 10  2 
\end{verbatim}
As a note, the paper states on page 32 that there are 37 unknown samples while the table on pages 33 and 34 shows that there are 34 unknowns.

%---------------------------------------------------------------------------------------------------------%
% - http://r.789695.n4.nabble.com/difference-between-createPartition-and-createfold-functions-td3865614.html

Using the data from the Examples section of \texttt{caret::createFolds}

\begin{framed}
\begin{verbatim}
library(caret) 
data(oil) 
part <- createDataPartition(oilType, 2) 
fold <- createFolds(oilType, 2) 

length(Reduce(intersect, part)) 
# [1] 27 
length(Reduce(intersect, fold)) 
#[1] 0 

\end{verbatim}
\end{framed}




\newpage
Looks like \texttt{createDataPartition} split your data into smaller pieces, 
but allows for the same example to appear in different splits. 

\texttt{createFolds} doesn't allow different examples to appear in different 
splits of the folds. 

%---------------------------------------------------------------------------------------------------------%

Basically, createDataPartition is used when you need to make one or 
more simple two-way splits of your data. For example, if you want to 
make a training and test set and keep your classes balanced, this is 
what you could use. It can also make multiple splits of this kind (or 
leave-group-out CV aka Monte Carlos CV aka repeated training test 
splits). 

createFolds is exclusively for k-fold CV. Their usage is simular when 
you use the returnTrain = TRUE option in createFolds. 

%- http://rgm3.lab.nig.ac.jp/RGM/R_rdfile?f=caret/man/downSample.Rd&d=R_CC
\newpage
\subsection{\texttt{createDataPartition}}
A series of test/training partitions are created using createDataPartition while createResample creates one or more bootstrap samples. createFolds splits the data into k groups while createTimeSlices creates cross-validation sample information to be used with time series data.
\begin{framed}
\begin{verbatim}
data(oil)
createDataPartition(oilType, 2)
\end{verbatim}
\end{framed}
\begin{verbatim}

$Resample1
 [1]  4  8  9 10 11 12 13 15 18 36 37 41 64 65 70 71 72 73 75
[20] 19 21 22 33 34 35 76 78 79 80 81 86 87 29 62 42 52 55 56
[39] 25 26 44 48 49 51 28 59 91 93 94 92

$Resample2
 [1]  4  6  7  9 10 12 13 14 16 18 37 38 40 64 68 69 72 73 75
[20] 19 20 21 23 24 32 33 35 81 83 84 87 88 29 62 42 53 55 56
[39] 25 26 47 49 50 96 58 59 60 93 95 27
\end{verbatim}

\begin{verbatim}
> createResample(oilType, 2)
$Resample1
 [1]  2  3  5  5  6  6  6  6  7  8  9  9  9  9 10 10 10 11 12
[20] 13 14 16 16 19 20 21 21 22 22 23 27 27 27 28 28 29 30 30
[39] 31 34 34 35 35 35 35 36 37 38 39 41 41 41 43 45 46 48 51
[58] 53 54 56 57 58 58 59 65 65 65 66 69 71 73 73 73 75 79 81
[77] 81 82 83 83 84 84 84 84 85 85 85 86 88 88 89 90 90 94 96
[96] 96

$Resample2
 [1]  4  4  6  6  8 10 11 11 11 12 14 16 16 17 17 17 19 20 20
[20] 20 21 22 22 22 24 26 27 28 30 31 33 34 34 35 37 38 41 41
[39] 43 44 45 45 47 47 54 55 56 57 57 59 59 60 60 61 63 63 65
[58] 65 65 65 66 66 67 68 68 69 69 70 74 77 77 77 77 78 78 79
[77] 80 81 82 82 82 83 83 84 85 85 86 87 87 88 89 90 91 93 94
[96] 94

\end{verbatim}


\end{document}
