\documentclass[a4paper,12pt]{article}
%%%%%%%%%%%%%%%%%%%%%%%%%%%%%%%%%%%%%%%%%%%%%%%%%%%%%%%%%%%%%%%%%%%%%%%%%%%%%%%%%%%%%%%%%%%%%%%%%%%%%%%%%%%%%%%%%%%%%%%%%%%%%%%%%%%%%%%%%%%%%%%%%%%%%%%%%%%%%%%%%%%%%%%%%%%%%%%%%%%%%%%%%%%%%%%%%%%%%%%%%%%%%%%%%%%%%%%%%%%%%%%%%%%%%%%%%%%%%%%%%%%%%%%%%%%%
\usepackage{eurosym}
\usepackage{vmargin}
\usepackage{amsmath}
\usepackage{graphics}
\usepackage{epsfig}
\usepackage{framed}
\usepackage{subfigure}
\usepackage{fancyhdr}

\setcounter{MaxMatrixCols}{10}
%TCIDATA{OutputFilter=LATEX.DLL}
%TCIDATA{Version=5.00.0.2570}
%TCIDATA{<META NAME="SaveForMode"CONTENT="1">}
%TCIDATA{LastRevised=Wednesday, February 23, 201113:24:34}
%TCIDATA{<META NAME="GraphicsSave" CONTENT="32">}
%TCIDATA{Language=American English}

\pagestyle{fancy}
\setmarginsrb{20mm}{0mm}{20mm}{25mm}{12mm}{11mm}{0mm}{11mm}
\lhead{MA4128} \rhead{Kevin O'Brien} \chead{Modelling Data} %\input{tcilatex}

%http://www.electronics.dit.ie/staff/ysemenova/Opto2/CO_IntroLab.pdf
\begin{document}


\section*{Theoretical Aspects of Fitting Models}

\section*{Ockham's razor and the Law of Parsimony}
\begin{itemize}
	\item Ockham's razor, sometimes known as the law of parsimony, is simply a maxim that states that simple explanations are usually better than complicated ones. \textbf{Ockham's razor} was originally proposed by a monk named William of Ockham. (He did not call it "Ockham's razor" or even "my razor." This is a name that has been given to it over time.)
	
\item Another version of this principle is the Law of parsimony . This says that if you are choosing between two theories, choose the one with the fewest assumptions. Assumptions here means claims of fact that have no evidence.
\item A theory that doesn't have many assumptions, and is very simple, is called a \textbf{parsimonious theory}.
	
\item In the context of statistics, the law of parsimony can be interpreted as an adequate model which requires the fewest independent variables is the preferred model.
\end{itemize}

\begin{framed}
	Parsimonious: The simplest plausible model with the fewest possible number of variables.
\end{framed}


\end{document}
