\documentclass[caret-main.tex]{subfiles}
\begin{document}
\section{Sonar Data Set}
The Sonar data consist of 208 data points collected on 60 predictors. 
The goal is to predict the two classes
%----------------------------%
The Sonar data consist of 208 data points collected on 60 predictors. 
The goal is to predict the two classes (M for metal cylinder or R for rock).
\begin{itemize}
\item[\textbf{M}] Metal cylinder 
\item[\textbf{R}] Rock
\end{itemize}

First, we split the data into two groups: a training set and a test set. To do this, the \texttt{createDataPartition} function is used



\begin{framed}
\begin{verbatim}
library(caret)
library(mlbench)
data(Sonar)
set.seed(107)
\end{verbatim}
\end{framed}

\begin{verbatim}
> head(Sonar)
      V1     V2     V3     V4     V5     V6     V7     V8
1 0.0200 0.0371 0.0428 0.0207 0.0954 0.0986 0.1539 0.1601
2 0.0453 0.0523 0.0843 0.0689 0.1183 0.2583 0.2156 0.3481
3 0.0262 0.0582 0.1099 0.1083 0.0974 0.2280 0.2431 0.3771
4 0.0100 0.0171 0.0623 0.0205 0.0205 0.0368 0.1098 0.1276
5 0.0762 0.0666 0.0481 0.0394 0.0590 0.0649 0.1209 0.2467
6 0.0286 0.0453 0.0277 0.0174 0.0384 0.0990 0.1201 0.1833
\end{verbatim}
\subsection{Partial least squares}
Partial least squares regression (PLS regression) is a statistical method that bears some relation to principal components regression; instead of finding hyperplanes of minimum variance between the response and independent variables, it finds a linear regression model by projecting the predicted variables and the observable variables to a new space. Because both the X and Y data are projected to new spaces, the PLS family of methods are known as bilinear factor models. Partial least squares Discriminant Analysis (PLS-DA) is a variant used when the Y is categorical.

\begin{framed}
\begin{verbatim}
training <- Sonar[ inTrain,]
testing <- Sonar[-inTrain,]
nrow(training)
nrow(testing)

plsFit <- train(Class ~ .,
 data = training,
 method = "pls",
 ## Center and scale the predictors for the training
 ## set and all future samples.
 preProc = c("center", "scale"))

\end{verbatim}
\end{framed}
\newpage
\begin{framed}
\begin{verbatim}
ctrl <- trainControl(method = "repeatedcv", repeats = 3)
plsFit <- train(Class ~ .,
	data = training,
	method = "pls",
	tuneLength = 15,
	trControl = ctrl,
	preProc = c("center", "scale"))
\end{verbatim}
\end{framed}

\end{document}
