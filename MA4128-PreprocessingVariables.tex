\documentclass[]{report}

\voffset=-1.5cm
\oddsidemargin=0.0cm
\textwidth = 480pt

\usepackage{framed}
\usepackage{subfiles}
\usepackage{enumerate}
\usepackage{graphics}
\usepackage{newlfont}
\usepackage{eurosym}
\usepackage{amsmath,amsthm,amsfonts}
\usepackage{amsmath}
\usepackage{color}
\usepackage{amssymb}
\usepackage{multicol}
\usepackage[dvipsnames]{xcolor}
\usepackage{graphicx}
\begin{document}



\subsection*{Introduction}
\begin{itemize}
\item
In statistics, the occurrence of several  variables in a multiple regression model are \textbf{closely correlated} to one another, and carrying the same information, more or less. Multi-collinearity can cause strange results when attempting to study how well individual independent variables contribute to an understanding of the dependent variable, often undermining the analysis.

\item In many analysis tasks, the variables under consideration are measured on
different scales or levels. This would
clearly distort any clustering analysis results. We can resolve this problem by \textbf{\textit{standardizing}}
the data prior to the analysis.

\item Different standardization methods are available, such as the simple \textbf{\textit{z standardization}},
which re-scales each variable to have a mean of 0 and a standard deviation of 1.

\item In most situations, however, \textbf{\textit{standardization by range}}(e.g., to a
range of 0 to 1 or -1 to 1) is preferable. We recommend standardizing the data
in general, even though this procedure can potentially reduce or inflate the variables’ influence
on the clustering solution.
\end{itemize}
%----------------------------------%
\section*{Transform Values}
The following alternatives are available for transforming values:
\begin{itemize}
	\item 	Z scores. Values are standardized to z scores, with a mean of 0 and a standard deviation of 1. 
	\item 	Range -1 to 1. Each value for the item being standardized is divided by the range of the values. 
	\item 	Range 0 to 1. The procedure subtracts the minimum value from each item being standardized and then divides by the range. 
	\item 	Maximum magnitude of 1. The procedure divides each value for the item being standardized by the maximum of the values. 
	\item 	Mean of 1. The procedure divides each value for the item being standardized by the mean of the values. 
	\item 	Standard deviation of 1. The procedure divides each value for the variable or case being standardized by the standard deviation of the values.
%	\item Additionally, you can choose how standardization is done. Alternatives are By variable or By case.
\end{itemize}
%%%%%%%%%%%%%%%%%%%%%%%%%%%%%%%%%%%%%%%%%%%%%%%%%%%%%%%%%%%%%%%%%%%%%%%%%%%%%%%%%%%%%%%%%%%%%%%%%555

\section*{Standardizing the Variables}
% % Moved to CA Notes
\begin{itemize}
	\item If variables are measured on different scales, variables with large values contribute
	more to the distance measure than variables with small values. In this example, both
	variables are measured on the same scale, so that’s not much of a problem, assuming
	the judges use the scales similarly. 
	\item But if you were looking at the distance between two
	people based on their IQs and incomes in dollars, you would probably find that the
	differences in incomes would dominate any distance measures. (A difference of only
	\$100 when squared becomes 10,000, while a difference of 30 IQ points would be only
	900. I’d go for the IQ points over the dollars!).
	
	\item Variables that are measured in large numbers will contribute to the distance more than variables recorded in smaller
	numbers.
	
	\item In the hierarchical clustering procedure in SPSS, you can standardize variables in
	different ways. You can compute standardized scores or divide by just the standard
	deviation, range, mean, or maximum. 
	\item This results in all variables contributing more
	equally to the distance measurement. That’s not necessarily always the best strategy,
	since variability of a measure can provide useful information. 
\end{itemize}


\end{document}
