\documentclass[a4]{beamer}
\usepackage{amssymb}
\usepackage{graphicx}
\usepackage{subfigure}
\usepackage{newlfont}
\usepackage{amsmath,amsthm,amsfonts}
%\usepackage{beamerthemesplit}
\usepackage{pgf,pgfarrows,pgfnodes,pgfautomata,pgfheaps,pgfshade}
\usepackage{mathptmx}  % Font Family
\usepackage{helvet}   % Font Family
\usepackage{color}

\mode<presentation> {
 \usetheme{Default} % was Frankfurt
 \useinnertheme{rounded}
 \useoutertheme{infolines}
 \usefonttheme{serif}
 %\usecolortheme{wolverine}
% \usecolortheme{rose}
\usefonttheme{structurebold}
}

\setbeamercovered{dynamic}

\title[MA4413]{MA4413 Statistics for Computing \\ {\normalsize MA4413 Lecture 7A : Review}}
\author[Kevin O'Brien]{Kevin O'Brien \\ {\scriptsize kevin.obrien@ul.ie}}
\date{Autumn 2011}
\institute[Maths \& Stats]{Dept. of Mathematics \& Statistics, \\ University \textit{of} Limerick}


\renewcommand{\arraystretch}{1.5}
%------------------------------------------------------------------------%


% - Sample Space (ready)
% - Conditional Probability
% - Independent Events (ready)
% - Expected Value (ready)
% - Variance V(X) (ready)
% - Combinations (ready)
% - Binomial distribution (ready)
% - Poisson distribution (ready)
% - MB 3 Ready
% - Standardization Formula (Normal Distribution) (ready)
% - Complement and Symmetry (ready)

\begin{document}
\begin{frame}
\titlepage
\end{frame}

%-------------------------------------------------------%
\frame{
\frametitle{Sample Space (1)}

\begin{itemize}
\item The sample space is the set of all possible outcomes of an experiment.
\item It is commonly denoted $S$.
\item The members of $S$ are called sample points.
\item A subset of $S$ is called an event.
\end{itemize}
}
%-------------------------------------------------------%
\frame{
\frametitle{Sample Space (2)}
Consider a random experiment in which a coin is tossed once, and a number between 1 and 4 is selected at random.
Write out the sample space $S$ for this experiment.

\bigskip

\[ S = \{(H,1),(H,2),(H,3),(H,4),(T,1),(T,2),(T,3),(T,4)\} \]

( $H$ and $T$ denoted `Heads' and `Tails' respectively. )

}
%-------------------------------------------------------%
\frame{
\frametitle{Contingency Tables}
Suppose there are 100 students in a first year college intake.  \begin{itemize} \item 44 are male and are studying computer science, \item 18 are male and studying statistics \item 16 are female and studying computer science, \item 22 are female and studying statistics. \end{itemize}
\bigskip
We assign the names $M$, $F$, $C$ and $S$ to the events that a student, randomly selected from this group, is male, female, studying computer science, and studying statistics respectively.
}
%-------------------------------------------------------%
\frame{
\frametitle{Contingency Tables}
The most effective way to handle this data is to draw up a table. We call this a \textbf{\emph{contingency table}}.
\\A contingency table is a table in which all possible events (or outcomes) for one variable are listed as
row headings, all possible events for a second variable are listed as column headings, and the value entered in
each cell of the table is the frequency of each joint occurrence.


\begin{center}
\begin{tabular}{|c|c|c|c|}
  \hline
  % after \\: \hline or \cline{col1-col2} \cline{col3-col4} ...
    & C & S & Total \\ \hline
  M & 44 & 18 & 62 \\ \hline
  F & 16 & 22 & 38 \\ \hline
  Total & 60 & 40 & 100 \\ \hline
\end{tabular}
\end{center}

}
%-------------------------------------------------------%
\frame{
\frametitle{Contingency Tables}
It is now easy to deduce the probabilities of the respective events, by looking at the totals for each row and column.
\begin{itemize}
\item P(C) = 60/100 = 0.60
\item P(S) = 40/100 = 0.40
\item P(M) = 62/100 = 0.62
\item P(F) = 38/100 = 0.38
\end{itemize}
\textbf{Remark:}\\
The information we were originally given can also be expressed as:
\begin{itemize}
\item $P(C \cap M) = 44/100 = 0.44$
\item $P(C \cap F) = 16/100 = 0.16$
\item $P(S \cap M) = 18/100 = 0.18$
\item $P(S \cap F) = 22/100 = 0.22$
\end{itemize}
}

%-------------------------------------------------------%
\frame{
\frametitle{Conditional Probability (1)}

The definition of conditional probability:
\[ P(A|B) = \frac{P(A \cap B)}{P(B)} \]

\begin{itemize}
\item $P(B)$ Probability of event B.
\item [ $P(A)$ Probability of event A. ] 
\item $P(A|B)$ Probability of event A given that B has occurred.
\item $P(A \cap B)$ Joint Probability of event A and event B.
\item Will be given tomorrow.
\end{itemize}

}


%-------------------------------------------------------%
\frame{
\frametitle{Conditional Probabilities (2)}

Compute the following:
\begin{enumerate}
\item $P(C|M)$ : Probability that a student is a computer science student, given that he is male.
\item $P(S|M)$ : Probability that a student studies statistics, given that he is male.
\item $P(F|S)$ : Probability that a student is female, given that she studies statistics.
\end{enumerate}

}
%-------------------------------------------------------%
\frame{
\frametitle{Conditional Probabilities (3)}

\textbf{Part 1)} Probability that a student is a computer science student, given that he is male.
\[ P(C|M) = \frac{P(C \cap M)}{P(M)}  = \frac{0.44}{0.62} = 0.71 \]
\textbf{Part 2)} Probability that a student studies statistics, given that he is male.
\[ P(S|M) = \frac{P(S \cap M)}{P(M)}  = \frac{0.18}{0.62} = 0.29 \]

}

%-------------------------------------------------------%
\frame{
\frametitle{Conditional Probabilities (4)}

\textbf{Part 3)} Probability that a student is female, given that she studies statistics.
\[ P(F|S) = \frac{P(F \cap S)}{P(S)}  = \frac{0.22}{0.40} = 0.55 \]




}
%------------------------------------------------------------%
\frame{
\frametitle{Bayes' Theorem}
Bayes' Theorem is a result that allows new information to be used to update the conditional probability of an event.
\bigskip

\[ P(A|B) = \frac{P(B|A)\times P(A)}{P(B)} \]

Use this theorem to compue $P(S|F)$, the probability that a student studies statistics, given that she is female.

\[ P(S|F) = \frac{P(F|S)\times P(S)}{P(F)} = {0.55 \times 0.40 \over 0.38} = 0.578\]
}

%-------------------------------------------------------%
\frame{
\frametitle{Independent Events}
\begin{itemize}
\item Suppose that a man and a woman each have a pack of 52 playing cards. 
\item Each draws a card from his/her pack. Find the probability that they each draw a Queen.
\item We define the events:
\begin{itemize} \normalsize \item A = probability that man draws a Queen = 4/52  = 1/13
\item B = probability that woman draws a Queen = 1/13
\end{itemize} \item Clearly events A and B are independent so:
\[ P(A \cap B) = 1/13 \times 1/13 = 0.005917 \]
\end{itemize}

}
%---------------------------------------------------------READY---------%
\frame{
\frametitle{Expected Value and Variance of a Random Variable}

The probability distribution of a discrete random variable is be tabulated as follows

\begin{center}
\begin{tabular}{|c||c|c|c|c|c|c|}
\hline
$x_i$  & 1 & 2 & 3 & 4 & 5 & 6 \\\hline
$p(x_i)$ & 2/8 & 1/8& 1/8 & 1/8& c & 1/8\\
\hline
\end{tabular}
\end{center}

\begin{itemize}
\item What is the value of $c$?
\item What is expected value and variance of the outcomes?
\end{itemize}
}
%---------------------------------------------------------READY---------%
\frame{
\frametitle{Expected value(1)}
\begin{itemize}
\item Necessarily $C =0.25 = 2/8$. \\
\item We must compute $E(X)$ as follows \[E(X) = \sum x_i p(x_i) \]
\item That formula is \textbf{not} given in the formulae.
\end{itemize}
\bigskip
$E(X) = (1 \times {2\over8}) + (2 \times {1 \over 8}) +  \ldots + (5 \times {2 \over 8}) + (6 \times {1 \over 8})$\\\bigskip
$E(X) = 27/8 = 3.375$\bigskip
}
%------------------------------------------------------READY------------%
\frame{
\frametitle{Variance(1)}
\begin{itemize}
\item The formula for computing the variance of a discrete random variable

\[ V(X) = E(X^2) - E(X)^2 \]

\item This is not given in the formulae for tomorrow's exam.

\item We must compute $E(X^2)$
\end{itemize}

\begin{center}
\begin{tabular}{|c||c|c|c|c|c|c|}
\hline
$x_i$  & 1 & 2 & 3 & 4 & 5 & 6 \\\hline
$x^2_i$  & 1 & 4 & 9 & 16 & 25 & 36 \\\hline
$p(x_i)$ & 2/8 & 1/8& 1/8 & 1/8& 2/8 & 1/8\\
\hline
\end{tabular}
\end{center}
}
%-----------------------------------------------------READY-------------%
\frame{
\frametitle{Variance (2)}

\begin{itemize}
\item $E(X^2) = (1 \times {2\over8}) + (4 \times {1 \over 8}) +  \ldots + (25 \times {2 \over 8}) + (36 \times {1 \over 8})$\bigskip
\item $E(X^2) = {117 \over 8} = 14.625$ \bigskip
\item $V(X) = E(X^2) - E(X)^2 = 14.625 - (3.375)^2 = 3.2344$
\end{itemize}
\bigskip

}

%---------------------------------------------------%
\begin{frame}
\frametitle{Combinations (1)}
Combinations formula
\[ ^{n}C_k  = {n! \over k!  \times (n-k)!} \]

\begin{itemize}
\item Remark $n! = n \times (n-1)! $
\item 0! = 1
\end{itemize}
\end{frame}
%---------------------------------------------------%
\begin{frame}
\frametitle{Combinations (2)}
Show that
\[ ^{n}C_0  = 1 \]

\textbf{Solution: }
\[ ^{n}C_0  = {n! \over 0!  \times (n-0)!} =  {n! \over n!} = 1 \]

\end{frame}
%---------------------------------------------------%
\begin{frame}
\frametitle{Combinations (3)}
Show that
\[ ^{n}C_1  = n \]

\textbf{Solution: }
\[ ^{n}C_1  = {n! \over 1!  \times (n-1)!} =  {n \times (n-1)! \over (n-1)!} = n \]

\end{frame}

%---------------------------------------------------%
\begin{frame}
\frametitle{Combinations (4)}
Compute $ ^{7}C_2  $\\

\textbf{Solution: }
\[ ^{7}C_2  = {7! \over 2!  \times (7-2)!} =  {7 \times 6 \times 5! \over 2! \times 5!} = {42 \over 2} =21  \]

\end{frame}
%---------------------------------------------------%
\begin{frame}
\frametitle{Combinations (5)}
Compute $ ^{11}C_1  $\\

\textbf{Solution: }
\[ ^{11}C_1  = {11! \over 1!  \times 10!} =  {11 \times 10! \over 1 \times 10!} = 11 \]

\end{frame}
%--------------------------------------------------------------------------------------%
\frame{
\frametitle{Binomial Distribution (1)}
\begin{itemize}
\item Identify the event that can considered the `success'.
\item (Remark : The success is usually the less likely of two complementary events.)
\item Determine the probability of a success in a single trial $p$.
\item Determine the number of independent trials $n$.
\end{itemize}

}
%--------------------------------------------------------------------------------------%
\frame{
\frametitle{Binomial Distribution (2)}
The probability of exactly k successes in a binomial experiment B(n, p) is given by
\[ P(X=k) = P(k \mbox{ successes }) = \;^nC_k  \times p^{k} \times (1-p)^{n-k}\]
Remark: This formula will be given tomorrow.
}

%--------------------------------------------------------------------------------------%
\frame{
\frametitle{Binomial Distribution (3)}

\begin{itemize}

\item Suppose we have a biased coin which yields a head only $48\%$ of the time.
\item Is this a binomial experiment?  why? 
\item What is the probability of 4 heads in 7 throws?
\end{itemize}


}
%--------------------------------------------------------------------------------------%
\frame{
\frametitle{Binomial Distribution (4)}

\begin{itemize}
\item X: Number of heads thrown
\item $n$ : number of independent trials (i.e. 7)
\item $k$ : Number of successes (numeric value)
\begin{itemize}
\item Here $k$ is 4 
\item Number of failures is $n-k  =3$
\end{itemize}
\item $p$ : probability of a success. (i.e. 0.48)
\item $1-p$ : probability of a failure (i.e. 0.52)
\end{itemize}

}
%--------------------------------------------------------------------------------------%
\frame{
\frametitle{Binomial Distribution (5)}

\[ P(X=4) = P(4 \mbox{ successes }) = \;^7C_4  \times (0.48)^{4} \times (0.52)^{3}\]

\bigskip

\[ P(X=4) = 35 \times 0.05308 \times  0.14061 =  \alert{0.2612} \]

Remark : must show workings.
}

%---------------------------------------------------------------------------%
\frame{
\frametitle{Poisson Distribution(1)}
The probability that there will be $k$ occurrences in a \textbf{unit time period} is denoted $P(X=k)$, and is computed as follows.
\Large
\[ P(X = k)=\frac{m^k e^{-m}}{k!} \]
\normalsize
This formula will be given tomorrow.
}
%---------------------------------------------------------------------------%
\frame{
\frametitle{Poisson Distribution(2)}
Given that there is on average 4 occurrences per day, what is the probability of one occurrences in a given day? \\ i.e. Compute $P(X=1)$ given that $m=4$
\Large
\[ P(X = 1)=\frac{4^1 e^{-4}}{1!} \]
\normalsize

The equation reduces to
\[ P(X = 1)=4 \times e^{-4} = \alert{0.07326} \]
}
%---------------------------------------------------------------------------%
\frame{
\frametitle{Poisson Distribution(3)}
What is the probability of one occurrences in a six hour period ? \\ i.e. Compute $P(X=1)$ given that $m=1$
\Large
\[ P(X = 1)=\frac{1^1 e^{-1}}{1!} \]
\normalsize
\begin{itemize}

\item $1!$ = 1
\end{itemize}
The equation reduces to
\[ P(X = 1) = e^{-1} = \alert{0.3678}\]
}
%------------------------------------------------------------------------%
\frame{
\begin{table}[ht]
\frametitle{Find $ P(Z \geq 1.27)$}
\vspace{-1.5cm}
%\caption{Standard Normal Distribution } % title of Table
\centering % used for centering table
\begin{tabular}{|c|| c c c c c c|} % centered columns (4 columns)
\hline %inserts double horizontal lines
& \ldots & \ldots & 0.06 &0.07&0.08&0.09 \\
%heading
\hline \hline% inserts single horizontal line
\ldots & \ldots & \ldots &\ldots& \ldots &\ldots&\dots \\ % inserting body of the table
1.0 & \ldots & \ldots &0.1446& 0.1423 &0.1401&0.1379 \\ % inserting body of the table
1.1 & \ldots & \ldots&0.1230& 0.1210 &0.1190&0.1170 \\ % inserting body of the table
1.2 & \ldots & \ldots&0.1038 & \alert{0.1020} & 0.1003&0.0985\\
1.3 & \ldots & \ldots &0.0869& 0.0853 &0.0838&0.0823 \\ % inserting body of the table
\ldots & \ldots &\ldots&\ldots & \ldots &\ldots&\ldots\\
\hline %inserts single line
\end{tabular}
\end{table}

Remark : Murdoch Barnes Table 3 will be given in tomorrow's exam. 
}
%------------------------------------------------------------------------%
\frame{
\frametitle{The Standardization Formula}
\begin{itemize}
\item Suppose that mean $\mu = 105 $ and that standard deviation $\sigma = 8$.
\item What is the Z-score for $x_o = 117$?
\[
z_{117} = {x_o - \mu \over \sigma} = {117 - 105 \over 8} = {12 \over 8} = 1.5
\]
\item Therefore $z_{117} = 1.5$
\item Remark: $P(X \geq 117) = P(Z \geq 1.5)$.
\end{itemize}
}
%-----------------------------------------------------%
\begin{frame}
\frametitle{Complement and Symmetry Rules}
\begin{itemize}
\item \textbf{Complement Rule}: \[ P(Z \leq k) = 1-P(Z \geq k) \] for some value $k$
\item Alternatively $ P(Z \geq k) = 1-P(Z \leq k) $
\item \textbf{Symmetry Rule}: \[ P(Z \leq -k) = P(Z \geq k) \] for some value $k$
\item Alternatively $ P(Z \geq -k) = P(Z \leq k) $
\end{itemize}
\end{frame}

%-----------------------------------------------------%
\begin{frame}
\frametitle{Complement and Symmetry Rules}
\textbf{Complement Rule} 

\begin{itemize}
\item $P(Z \leq 1.27) = 1-P(Z \geq 1.27) $ 
\item $ P(Z \geq 1.27) = 1-P(Z \leq 1.27) $
\end{itemize}

\bigskip 
\textbf{Symmetry Rule}: 
\begin{itemize}
\item $ P(Z \leq -1.27) = P(Z \geq 1.27) $
\item $ P(Z \geq -1.27) = P(Z \leq 1.27) $
\end{itemize}

Complement rule and Symmetry rule can be used in conjunction.
\end{frame}




%-----------------------------------------------------%
\begin{frame}
\frametitle{Complement and Symmetry Rules}

For a normally distributed random variable with mean $\mu = 1000$ and standard deviation $\sigma = 100$, compute $P(X \geq 873)$.

\begin{itemize} \item First, find the Z-value using the standardization formula.
\[
z_{873} = {x_o - \mu \over \sigma} = {873 - 1000 \over 100} = {-127 \over 100} = -1.27
\]
\item We can say $P(X \geq 873) = P(Z \geq -1.27)$.
\item Use complement rule and symmetry rule to evaluate  $P(Z \geq -1.27)$.
\item $ P(Z \geq -1.27) = P(Z \leq 1.27) = 1 - P(Z \geq 1.27) $  = 1 - 0.1020 = \alert{0.8980}.
\end{itemize}
\end{frame}
%------------------------------------------------------------------------%


%-----------------------------------------------------%
\begin{frame}
\frametitle{Interval Probability}
\begin{itemize}
\item We are often interested in the probability of being inside an interval, with lower bound $L$ and upper bound $U$.
\item It is often easier to compute the probability of the complement event, being outside the interval.
\[ P( \mbox{Inside} ) = 1 - P( \mbox{Outside} )  \]

\item Being outside the interval is the conjunction of being too low and too high.
\[ P( \mbox{Outside} ) = P( \mbox{Too Low} ) +  P( \mbox{Too High} ) \]

\item Therefore we can say
\[ P( \mbox{Inside} ) = 1- [P( \mbox{Too Low} ) +  P( \mbox{Too High} )] \]
\item $P( \mbox{Too Low} )$ = $P( X \leq L)$
\item $P( \mbox{Too High} )$ = $P( X \geq U)$
\end{itemize}
\end{frame}
\end{document}




