
\documentclass[a4]{beamer}
\usepackage{amssymb}
\usepackage{graphicx}
\usepackage{subfigure}
\usepackage{newlfont}
\usepackage{amsmath,amsthm,amsfonts}
%\usepackage{beamerthemesplit}
\usepackage{pgf,pgfarrows,pgfnodes,pgfautomata,pgfheaps,pgfshade}
\usepackage{mathptmx}  % Font Family
\usepackage{helvet}   % Font Family
\usepackage{color}

\mode<presentation> {
 \usetheme{Default} % was Frankfurt
 \useinnertheme{rounded}
 \useoutertheme{infolines}
 \usefonttheme{serif}
 %\usecolortheme{wolverine}
% \usecolortheme{rose}
\usefonttheme{structurebold}
}

\setbeamercovered{dynamic}

\title[MathsCast]{MathsCast Presentations \\ {\normalsize Lecture 6B}}
\author[Kevin O'Brien]{Kevin O'Brien \\ {\scriptsize kevin.obrien@ul.ie}}
\date{Summer 2011}
\institute[Maths \& Stats]{Dept. of Mathematics \& Statistics, \\ University \textit{of} Limerick}

\renewcommand{\arraystretch}{1.5}

\begin{document}
\begin{frame}
\titlepage
\end{frame}



%----------------------------------------------------%
\begin{frame}
\frametitle{Statistical Inference : Estimation}
\begin{itemize}
\item When a parameter is being estimated, the estimate can be either a single number or it can be a range of numbers.
\item When the estimate is a single number, such as a sample mean, the estimate is called a \textbf{\emph{point estimate}}.
\item When the estimate is a range of values, the estimate is called an \textbf{\emph{interval estimate}}.
\item \textbf{\emph{Confidence intervals}} are used for interval estimation.
\item As we will soon see, point estimates are not usually as informative as confidence intervals.
\end{itemize}
\end{frame}


%----------------------------------------------------%
\begin{frame}
\frametitle{Statistical Inference : Confidence Intervals}
\begin{itemize}
\item Confidence intervals allow us to use sample data to estimate a parameter value, such as a population mean.
\item A confidence interval is a range of values for which we can be confident (at a specific level) that parameter value (such as the population mean)  lies within.
\item A confidence level will have a specified level of confidence, commonly $95\%$.
\item The $95\%$ confidence interval is a range of values which contains the parameter value of interest with a probability of 0.95.
\item We can expected that a $95\%$ confidence interval will not contain the parameter value of interest with a probability of 0.05.
\end{itemize}
\end{frame}


%----------------------------------------------------%
\begin{frame}
\frametitle{Statistical Inference : Confidence Intervals}
\begin{itemize}

\item It is natural to interpret a $95\%$ confidence interval on the mean as an interval with a 0.95 probability of containing the population mean.
\item However, the proper interpretation is not that simple.
\item Consider the case in which 1,000 studies estimating the value of $\mu$  in a certain population all resulted
in estimates between 30 and 40.
\item Suppose one more study was conducted and the $95\%$ confidence interval on $\mu$ was computed
to be $40 \leq \mu \leq 50$ (based on that one study).

\item The probability that $\mu$ is between 40 and 50 is very low, the confidence interval not withstanding.

\end{itemize}
\end{frame}

%----------------------------------------------------%
\begin{frame}
\frametitle{Central Limit Theorem}

\begin{itemize}

\item Before we can begin computing confidence intervals, we must introduce the \textbf{\emph{Central Limit Theorem}}.

\item Suppose random sample of size $n$ are drawn from any distribution, with the distribution having a mean of $\mu$ (equivalently $E(X)$) and variance of $\sigma^2$ (i.e. standard deviation of $\sigma$).

\item Also suppose that the sample size is large ( i.e. $n > 30$ ).

\item The sample means tend to form a normal distribution with mean $\mu$ and standard deviation $ { \sigma \over \sqrt{n} }$

\item We call the standard deviation of the sample means the \textbf{\emph{standard error}}
\item Standard error is commonly denoted as $S.E.$
\end{itemize}
\end{frame}

%----------------------------------------------------%
\begin{frame}
\frametitle{Central Limit Theorem}

\begin{itemize}

\item Recall from earlier lectures, an experiment was carried out where the sum of 100 throws of a die were recorded.

\item The underlying distribution of the die values is not normally distributed. \\(Actually discrete uniform between 1 and 6.)

\item Nonetheless the distribution of the sum of 100 throws was normally distributed. Necessarily the distribution of the average score for 100 throws is normally distributed.

\end{itemize}
\end{frame}

\frame{
\frametitle{Distribution of means}

\begin{center}
\includegraphics[scale=0.4]{6BHist}
\end{center}

}
%----------------------------------------------------%
\begin{frame}
\frametitle{Exercise}
From previous lecture, we know the following properties of the dice distribution. \\(Remark: In this case we know the variance, but that is not always the case.)
\begin{itemize}
\item Mean (Expected Value) $E(X) = \mu = 3.5$
\item Variance $V(X) = \sigma^2 = 2.9166$
\item Standard deviation $= \sigma = 1.707$
\end{itemize}

Compute the standard error $S.E.(\bar{x})$ for the mean value $\bar{x}$ of die values:
\begin{itemize}
\item when the die is thrown 25 times
\item when the die is thrown 225 times.
\end{itemize}
\end{frame}

%----------------------------------------------------%
\begin{frame}
\frametitle{Exercise}
\begin{itemize}
\item When the die is thrown 25 times n = 25
\item Therefore the standard error is
\[ {\sigma \over \sqrt{n}}  = {1.707 \over \sqrt{25}} = {1.707 \over 5} = 0.3415. \]
\item When the die is thrown 225 times: n = 225
\item Therefore the standard error is
\[ {\sigma \over \sqrt{n}}  = {1.707 \over \sqrt{225}} = {1.707 \over 15} = 0.1138. \]

\end{itemize}
\end{frame}
\frame{
\frametitle{Distribution of means}

\begin{center}
\includegraphics[scale=0.4]{6BHist2}
\end{center}

}

\frame{
\frametitle{Distribution of means}

\begin{center}
\includegraphics[scale=0.4]{6BHist3}
\end{center}

}

%----------------------------------------------------%
\begin{frame}
\frametitle{Exercise}
\begin{itemize}
\item Compare the two histograms on the previous slides. These horizontal range of value is the same for both histograms.

\item We can see that with a larger sample size ($n=225$), the distribution of sample means are clustered closely around the 3.5 mark, and have much less dispersion than distribution of sample means with a sample size $n=25$.

\end{itemize}
\end{frame}
\end{document}


